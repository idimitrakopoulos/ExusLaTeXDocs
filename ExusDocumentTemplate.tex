\documentclass[a4paper,10pt,oneside,table,pdftex,dvipsnames]{report}


%include packages
\usepackage{fontspec}
\usepackage{soul}
\usepackage{ExusCorporateTemplate}
\usepackage{ExusSpecificationsTemplate}


% Document Variables
\newcommand{\doctitle}{Document Title}
\newcommand{\version}{v0.1}
\newcommand{\exusdepartment}{Web \& Mobile Business Unit}


\begin{document}
\setmainfont{Calibri}


\addexusfrontpage{\exusdepartment}{\doctitle \space \version}{Customer}{Athens}

\addexusheaderfooter{\doctitle}{\exusdepartment}

\tableofcontents

\nl

This document has been officially approved for the specific project, and accurately reflects the current understanding of requirements based on the provided business, system functional \& non-functional requirements. 
\\\\
Following approval of this document, requirement changes will be governed by the project’s change management process, including impact analysis, appropriate reviews and approvals.





\chapter{Overview}
\section{Purpose}

This document outlines the Business, System, Functional \& Non-Functional requirements that will be used to develop the described solution. 
\\\\
The primary audience of this document are Business and Technical stakeholders in order to verify that their requirements have been documented accurately and completely.
\\\\
This will be the same document that will be passed on to development and quality assurance teams for implementation. It will be used as a reference for developing code, test plans, test cases and as a definition of done in order to verify that all mentioned requirements are met.
\\\\
\small \textit{© 2015 Exus. All rights reserved}

\section{Scope}

\hl{Briefly describe project’s functional (new features, improvements to existing features, integrations with other systems, migrations), and tangible deliverables (source code, documentation, test cases/plan, evidences).
\\\\
e.g. As a (user), I can (do something) so that (I receive some benefit)}


\chapter{Functional Requirements}
\section{Use Cases}

\subsection{UC1}

\begin{usecase}
	% --------------------------------------------------------------
	\ucid{						% ID
		ID
	} 							
	% --------------------------------------------------------------
	\ucversion{					% Version
		Version
	}
	% --------------------------------------------------------------
	\ucname{					% Name
		Name
	}
	% --------------------------------------------------------------
	\ucdescription{				% Description
		Description
	}
	% --------------------------------------------------------------	
	\ucactors{					% Actors
		\item Actor 1
		\item Actor 2
	}	
	% --------------------------------------------------------------
	\uctriggers{				% Triggers
		\item Trigger 1
		\item Trigger 2
	}	
	% --------------------------------------------------------------	
	\ucpreconditions{			% Preconditions
		\item Precondition 1
		\item Precondition 2
	}	
	% --------------------------------------------------------------
	\ucpostconditions{			% Postconditions
		\item Postcondition 1
		\item Postcondition 2
	}	
	% --------------------------------------------------------------	
	\ucmainflow{				% Main Flow
		\item This is the first action
		\item This is the second action
	}	
	% --------------------------------------------------------------	
	\ucextflows{				% Extension Flows
		\additemizedsubfield{5.a}		% Extension Flow ID
		{Ext Flow Name}					% Extension Flow Name
		{								% Extension Flow Steps
			\item[1.] Step 1
			\item[2.] Step 2
		}	
	}	
	% --------------------------------------------------------------
	\ucerrflows{				% Error Flows
		\additemizedsubfield{5.a}		% Error Flow ID
		{Ext Flow Name}					% Error Flow Name
		{								% Error Flow Steps
			\item[1.] Step 1
			\item[2.] Step 2
		}	
	}	
	% --------------------------------------------------------------	
	\ucincludes{				% Includes
		\item Included UC
	}	
	% --------------------------------------------------------------
	\ucspecialreqs{				% Special Requirements
		\item Item 1
	}		
	% --------------------------------------------------------------
	\ucassumptions{				% Assumptions
		\item Item 1
	}	
	% --------------------------------------------------------------	
	\ucnotes{					% Notes
		\item Item 1
	}	
	% --------------------------------------------------------------	
	\ucfutureconsiderations{	% Future Considerations
		\item Item 1
	}
	% --------------------------------------------------------------	

\end{usecase}




\chapter{System and Non-Functional Requirements}

\hli{System \& Non-Functional requirements refer to building the solution right (as opposed to the “right solution” from functional requirements). They always involve trade-offs. Meeting a single non-functional requirement can sometimes cause your app not meet another one.}

\section{Software Compatibility}

\hli{A list of Mobile OS or Browser compatibility}

\section{Hardware Compatibility}

\hli{A list of screen resolution \& screen form factor/ handset compatibility}

\section{Orientation Compatibility}

\hli{Portrait/Landscape}

\section{Lingual Compatibility}

\hli{Supported languages and text orientations}


\chapter{Project Execution}

\section{References}

\hli{List external/customer references and controlling documents, including: BRD documents, meeting summaries, white papers, API documentation, other deliverables, etc}


\section{Assumptions}

\hli{Provide a list of contractual or technical level assumptions and/or constraints that are preconditions to project completion. Assumptions are future situations beyond the control of the project, whose outcomes influence the success of a project. Examples of assumptions include: availability of a technical platform, supporting teams, test data, legal changes and policy decisions.}

\section{Constraints}

\hli{Constraints are boundary conditions on how the system must be designed and constructed. Examples include: technical standards, strategic decisions, legal requirements.
Constraints exist because of real business conditions. For example, a delivery date is a constraint only if there are real business consequences that will happen as a result of not meeting the date. If failing to have the subject application operational by the specified date places the organization in legal default, the date is a constraint}


\section{Points of Contact and Stakeholders}

\hli{Describe the persons/departments of contact for the project}

\chapter{Appendix}

\section{Glossary and Acronyms}

This document contains acronyms and terminology relevant to the Web \& Mobile industry as well as technical nomenclature that is best described in the table below.
\\\\


\section{Attachments}

\hli{attachments}


\end{document}

