\documentclass[a4paper,10pt,oneside,table,pdftex,dvipsnames]{report}


%include packages
\usepackage{fontspec}
\usepackage{soul}
\usepackage{ExusCorporateTemplate}
\usepackage{ExusSpecificationsTemplate}


% Document Variables
\newcommand{\doctitle}{Document Title}
\newcommand{\version}{v0.1}
\newcommand{\exusdepartment}{Web \& Mobile Business Unit}


\begin{document}
\setmainfont{Calibri}


\addexusfrontpage{\exusdepartment}{\doctitle \space \version}{Customer}{Athens}

\addexusheaderfooter{\doctitle}{\exusdepartment}

\tableofcontents

\leavevmode \newline

This document has been officially approved for the specific project, and accurately reflects the current understanding of requirements based on the provided business, system functional \& non-functional requirements. 
\\\\
Following approval of this document, requirement changes will be governed by the project’s change management process, including impact analysis, appropriate reviews and approvals.





\chapter{Overview}
\section{Purpose}

This document outlines the Business, System, Functional \& Non-Functional requirements that will be used to develop the described solution. 
\\\\
The primary audience of this document are Business and Technical stakeholders in order to verify that their requirements have been documented accurately and completely.
\\\\
This will be the same document that will be passed on to development and quality assurance teams for implementation. It will be used as a reference for developing code, test plans, test cases and as a definition of done in order to verify that all mentioned requirements are met.


\section{Scope}

\hl{Briefly describe project’s functional (new features, improvements to existing features, integrations with other systems, migrations), and tangible deliverables (source code, documentation, test cases/plan, evidences).
\\\\
e.g. As a (user), I can (do something) so that (I receive some benefit)}


\chapter{Functional Requirements}
\section{Use Cases}

\subsection{UC1}


\begin{usecase}

\addtitle{Use Case:}{eTopup - Pay via Paypal} 

%Scope: the system under design
\addfield{ID:}{UC1}

%Level: "user-goal" or "subfunction"
\addfield{Level:}{User-goal}

%Primary Actor: Calls on the system to deliver its services.
\addfield{Primary Actor:}{End-User}

%Stakeholders and Interests: Who cares about this use case and what do they want?
\additemizedfield{Stakeholders and Interests:}{
	\item Stakeholder 1 name: his interests
	\item Stakeholder 2 name: his interests
}

%Preconditions: What must be true on start and worth telling the reader?
\addfield{Preconditions:}{}
%when multiple
%\additemizedfield{Preconditions:}{} 

%Postconditions: What must be true on successful completion and worth telling the reader
\addfield{Postconditions:}{}
%when multiple
%\additemizedfield{Preconditions:}{}

%Main Success Scenario: A typical, unconditional happy path scenario of success.
\addscenario{Main Success Scenario:}{
	\item \label{lalakis} The first action
	\item The second action
}

%Extensions: Alternate scenarios of success or failure.
\addscenario{Extensions:}{
	\item[2.a] Invalid login data:
		\begin{enumerate}
		\item[1.] System shows failure message
		\item[2.] User returns to step 1
		\end{enumerate}
	\item[5.a] Invalid subsriber data:
		\begin{enumerate}
		\item[1.] System shows failure message
		\item[2.] User returns to step 2 and corrects the errors
		\end{enumerate}
}

%Special Requirements: Related non-functional requirements.
\additemizedfield{Special Requirements:}{
	\item first applicable non-functional requirement
	\item second applicable non-functional requirement. see more in \ref{lalakis}
}

%Technology and Data Variations List: Varying I/O methods and data formats.
\addscenario{Technology and Data Variations List:}{
	\item[1a.] Alternative first action with other technology
}

%Frequency of Occurrence: Influences investigation, testing and timing of implementation.
\addfield{Frequency of Occurrence:}{}

%Miscellaneous: Such as open issues/questions
%\addfield{Open Issues:}{}

\end{usecase}


\end{document}

